% Options for packages loaded elsewhere
\PassOptionsToPackage{unicode}{hyperref}
\PassOptionsToPackage{hyphens}{url}
\PassOptionsToPackage{dvipsnames,svgnames,x11names}{xcolor}
%
\documentclass[
]{book}
\usepackage{amsmath,amssymb}
\usepackage{lmodern}
\usepackage{iftex}
\ifPDFTeX
  \usepackage[T1]{fontenc}
  \usepackage[utf8]{inputenc}
  \usepackage{textcomp} % provide euro and other symbols
\else % if luatex or xetex
  \usepackage{unicode-math}
  \defaultfontfeatures{Scale=MatchLowercase}
  \defaultfontfeatures[\rmfamily]{Ligatures=TeX,Scale=1}
\fi
% Use upquote if available, for straight quotes in verbatim environments
\IfFileExists{upquote.sty}{\usepackage{upquote}}{}
\IfFileExists{microtype.sty}{% use microtype if available
  \usepackage[]{microtype}
  \UseMicrotypeSet[protrusion]{basicmath} % disable protrusion for tt fonts
}{}
\makeatletter
\@ifundefined{KOMAClassName}{% if non-KOMA class
  \IfFileExists{parskip.sty}{%
    \usepackage{parskip}
  }{% else
    \setlength{\parindent}{0pt}
    \setlength{\parskip}{6pt plus 2pt minus 1pt}}
}{% if KOMA class
  \KOMAoptions{parskip=half}}
\makeatother
\usepackage{xcolor}
\usepackage{longtable,booktabs,array}
\usepackage{calc} % for calculating minipage widths
% Correct order of tables after \paragraph or \subparagraph
\usepackage{etoolbox}
\makeatletter
\patchcmd\longtable{\par}{\if@noskipsec\mbox{}\fi\par}{}{}
\makeatother
% Allow footnotes in longtable head/foot
\IfFileExists{footnotehyper.sty}{\usepackage{footnotehyper}}{\usepackage{footnote}}
\makesavenoteenv{longtable}
\usepackage{graphicx}
\makeatletter
\def\maxwidth{\ifdim\Gin@nat@width>\linewidth\linewidth\else\Gin@nat@width\fi}
\def\maxheight{\ifdim\Gin@nat@height>\textheight\textheight\else\Gin@nat@height\fi}
\makeatother
% Scale images if necessary, so that they will not overflow the page
% margins by default, and it is still possible to overwrite the defaults
% using explicit options in \includegraphics[width, height, ...]{}
\setkeys{Gin}{width=\maxwidth,height=\maxheight,keepaspectratio}
% Set default figure placement to htbp
\makeatletter
\def\fps@figure{htbp}
\makeatother
\setlength{\emergencystretch}{3em} % prevent overfull lines
\providecommand{\tightlist}{%
  \setlength{\itemsep}{0pt}\setlength{\parskip}{0pt}}
\setcounter{secnumdepth}{5}
\usepackage{booktabs}
\usepackage{amsthm}
\makeatletter
\def\thm@space@setup{%
  \thm@preskip=8pt plus 2pt minus 4pt
  \thm@postskip=\thm@preskip
}
\makeatother

\usepackage[a4paper]{geometry}
\geometry{left=2cm}
\geometry{right=2cm}
\geometry{bottom=2cm}
\geometry{top=2cm}

%\usepackage{hyperref}
%\hypersetup{
%    colorlinks=true,
%    linkcolor=blue,
%    filecolor=magenta,
%    urlcolor=cyan,
%    }

% https://tex.stackexchange.com/a/233271
\usepackage[explicit]{titlesec}
\titleformat{name=\section,numberless}[hang]{}{}{0cm}{%
  \LARGE #1\markboth{#1}{#1}%
}

\frontmatter
\ifLuaTeX
  \usepackage{selnolig}  % disable illegal ligatures
\fi
\usepackage[]{natbib}
\bibliographystyle{plainnat}
\IfFileExists{bookmark.sty}{\usepackage{bookmark}}{\usepackage{hyperref}}
\IfFileExists{xurl.sty}{\usepackage{xurl}}{} % add URL line breaks if available
\urlstyle{same} % disable monospaced font for URLs
\hypersetup{
  pdftitle={Doctrinal Declarations of the Presbytery},
  pdfauthor={Evangel Presbytery},
  colorlinks=true,
  linkcolor={Maroon},
  filecolor={Maroon},
  citecolor={Blue},
  urlcolor={Blue},
  pdfcreator={LaTeX via pandoc}}

\title{Doctrinal Declarations of the Presbytery}
\author{Evangel Presbytery}
\date{2025-09-02}

\begin{document}
\maketitle



{
\hypersetup{linkcolor=}
\setcounter{tocdepth}{1}
\tableofcontents
}
\hypertarget{preface}{%
\chapter*{Preface}\label{preface}}
\addcontentsline{toc}{chapter}{Preface}

This section of Evangel Presbytery's \href{https://bco.evangelpresbytery.com/preface.html\#c.-constitution-defined}{constitution} comprises statements addressing important matters of faith and practice today. These are matters of such doctrinal and moral weight that we believe they rise above the level of the governmental rules laid down in our \href{https://bco.evangelpresbytery.com/}{Book of Church Order}. They are more similar in their nature to the Westminster Standards, and they aspire to carry on the work of reform in the church of Jesus Christ by addressing errors and sins particular to our own day and age. Recognizing that these declarations are new and relatively untested, we thought it unwise and presumptuous to append them directly to, or to grant them the same constitutional weight as, the Westminster Standards. Therefore we have created these Doctrinal Declarations of the Presbytery, which currently include:

\begin{enumerate}
\def\labelenumi{\arabic{enumi}.}
\item
  Declaration of Doctrine on Sexuality
\item
  Declaration of Doctrine on Abortion
\end{enumerate}

You can always find the latest online version of these declarations at \url{https://ddp.evangelpresbytery.com} and download the \href{https://ddp.evangelpresbytery.com/evangel-presbytery-ddp.pdf}{latest PDF version here}. A record of the changes can be found in the \href{https://ddp.evangelpresbytery.com/updates.html}{Updates} section at the end of the book.

\mainmatter

\hypertarget{declaration-of-doctrine-on-sexuality}{%
\chapter*{Declaration of Doctrine on Sexuality}\label{declaration-of-doctrine-on-sexuality}}
\addcontentsline{toc}{chapter}{Declaration of Doctrine on Sexuality}

\hypertarget{introduction}{%
\section*{Introduction}\label{introduction}}
\addcontentsline{toc}{section}{Introduction}

Since the mid-twentieth century, rebellion against God's divine pattern of sex within the loving union of lifelong, monogamous, heterosexual marriage has become widespread, and attacks against that pattern are increasingly perpetrated in concert with the civil magistrate. Thus for the protection of the Christian church's conscience and the purity of the faith once for all delivered to the saints, it has become necessary for Christians to declare our scriptural convictions and commitments concerning sexuality and marriage. These convictions and commitments are testified to by God's natural revelation and they are explicitly commanded by God's special revelation in Holy Scripture. Prior generations of Christians and non-Christians alike lived under the beneficial constraints of laws written to protect civil society from sexual relations outside the commands of Scripture. Western law in particular reinforced God's law concerning sexual relations and His Creation Order of Adam first, then Eve. For this reason, the church had little need to adopt doctrinal creeds or statements concerning sexuality. Now though, with the heathens' attack growing ever more intense and becoming institutionalized by the power of civil authority, the time has come for the church to declare her allegiance to God's law of male and female, and to do so specifically, forthrightly, and with confidence in the wisdom and kindness of God. To that end, we declare our biblical convictions and commitments for the glory of Christ and the good of His church.

\hypertarget{on-the-order-of-creation}{%
\section*{On the Order of Creation}\label{on-the-order-of-creation}}
\addcontentsline{toc}{section}{On the Order of Creation}

\begin{enumerate}
\def\labelenumi{\arabic{enumi}.}
\item
  ``In the day when God created man, He made him in the likeness of God. God created them male and female, and He blessed them and named them Man in the day when they were created.''1 God formed the first male, Adam, from the dust of the ground.2 He made the first female, Eve, from Adam's rib and presented her to Adam to be a helper perfectly suited to him.3 Adam called his wife ``Woman, because she was taken out of Man.''4 God named the human race, consisting of both males and females, ``Man'' after the first man Adam, whose name is simply the Hebrew word for ``man'' used throughout the Old Testament.
\item
  From the beginning, God gave Adam authority over Eve and responsibility for her. This authority and responsibility are inseparably joined together, and they lay the foundation for God's decree of father-rule (patriarchy) throughout Scripture. Eve was created for Adam's sake5 to be a ``help meet'' for him, that is, a fitting helper.6 Man is to love and take responsibility for woman by leading her and laying down his life for her, providing a living illustration of Christ's sacrificial leadership of His Bride, the church.7
\item
  God's Creation Order of man and woman was established while man was in a state of innocence before the Fall and is therefore a universal rule for all mankind. God's creation of Adam first, then Eve, is the origin of Scripture's condemnation of woman exercising authority over man8 and is further elucidated by Scripture's declaration that man is the glory of God, but woman is the glory of man.9 This Creation Order is also the source of Scripture's commands that husbands love their wives as their own bodies,10 and that wives be subject to their husbands in everything.11
\item
  God's subordination of woman to man in no way diminishes woman's perfect equality with man in essence, worth, and honor.12
\end{enumerate}

\hypertarget{on-sexual-identity}{%
\section*{On Sexual Identity}\label{on-sexual-identity}}
\addcontentsline{toc}{section}{On Sexual Identity}

\begin{enumerate}
\def\labelenumi{\arabic{enumi}.}
\setcounter{enumi}{4}
\item
  God's bifurcation of mankind into two and only two sexes, male and female, is an act of His creative will and power and continues through all generations since Adam, our first father.
\item
  God forms each person in his mother's womb and creates each unborn child as either male or female.13 From the moment of conception,14 males are distinct from females, and females are distinct from males.15 What God has decreed as each one's sex at the moment of conception, either male or female, is His gift and must be received with gratitude, each man living out his manhood and each woman her womanhood, in humble reliance upon God's grace.16
\item
  The proper conception of ``sexual identity'' is a matter of living in accordance with the genetic sex given to us by God, either male or female. Genetic sex and sexual identity cannot be separated, and they remain bound together throughout one's life. Sexuality does not admit of gradations. You are either male or female, not part male and part female. Nor are there a great multitude of sexual identities. There are only two, male and female.
\item
  Any attempt of a man to play the woman or a woman to play the man violates God's decree, attacks His created order, and constitutes sin so serious that God Himself pronounces it an ``abomination.''17

  \begin{enumerate}
  \def\labelenumii{\alph{enumii}.}
  \tightlist
  \item
    This sin includes transvestism and any efforts, including chemical or surgical, as well as behavioral (e.g., effeminacy18), to reject and efface one's sex and to adopt characteristics of the opposite sex.\\
  \item
    This sin also includes the conscription of woman as a military combatant or the placing of woman in harm's way as a law enforcement officer. As life-giver,19 woman has always been honored by man's defense of her and her children. Furthermore, Christ Jesus gave up His life for His Bride, the church, and man is to follow Christ's example by laying down his life in defense of woman.20 A civil magistrate defaces woman's sexuality by placing her in the uniform of a combatant21 and commanding her to take up arms. Further, given woman's comparative physical weakness in the face of male enemies, such magistrates place their homeland at unnecessary risk.22 Finally, female military combatants of childbearing age often (whether knowingly or unknowingly) place at risk unborn children, which is an act contrary to the just war principle of avoiding needless loss of life.
  \end{enumerate}
\end{enumerate}

\hypertarget{on-marriage}{%
\section*{On Marriage}\label{on-marriage}}
\addcontentsline{toc}{section}{On Marriage}

\begin{enumerate}
\def\labelenumi{\arabic{enumi}.}
\setcounter{enumi}{8}
\item
  Marriage is instituted by God as the lifelong23 monogamous24 union of one male and one female.25 This relationship, with these limitations, was established while man was in a state of innocence before the Fall,26 and its pattern is established for all members of the human race. Marriage is an honorable estate that God Himself made, and it symbolizes to us the mystical union which is between Christ and His church.27
\item
  Just as Adam was given authority over Eve and responsibility for her, and just as Christ is the head of the church, so each husband is the head of his wife.28 God commands each husband to love his wife, ``just as Christ also loved the church and gave Himself up for her.''29
\item
  Just as Eve was created for Adam, and just as the church is subject to Christ, so each wife is to be subject to her own husband in everything.30
\item
  For centuries, Christians have recognized the following vital purposes of marriage:\\
  ``Marriage was ordained for the mutual help of husband and wife, for the increase of mankind with a legitimate issue, and of the Church with an holy seed; and for preventing of uncleanness.''31
\item
  When sin entered the human race, marriage was severely harmed.32 Wives began rebelling against their husbands' authority, husbands began abdicating and abusing their authority over their wives, and sexual immorality began to corrupt marital intimacy, all of which dishonor God and bring shame to the name of Christ and His Bride, the church. God therefore created laws to govern the violation of His established pattern, while not changing the pattern.33 Since marriage is a lifelong and monogamous union, God forbids divorce except in two circumstances: (1)~when one spouse abandons the other,34 and (2)~when a spouse engages in sexual immorality.35
\item
  God in His Law forbids any deviation from the established pattern of marriage and from His gift of sexual union to be enjoyed by a married couple. These forbidden deviations include lust (i.e., sexual desire for anyone but one's spouse), pornography,36 masturbation (including simulated copulation with any inanimate object, no matter how lifelike),37 fornication,38 adultery,39 polygamy,40 incest,41 pedophilia,42 homosexuality,43 and bestiality.44 Because the heart of man is deceitful above all things and desperately wicked,45 it is impossible to catalog fully all forms of sexual immorality and degradation.
\item
  Presently, there is widespread disregard, even scorn, for the divine standard of sexuality and marriage as revealed in Scripture. This disregard, which is found both in the world and in many churches, endangers the family (our basic social unit) and causes much suffering to the innocent, especially children. When found in the church, it brings shame to the name of Christ.46
\end{enumerate}

\hypertarget{on-faith-and-repentance}{%
\section*{On Faith and Repentance}\label{on-faith-and-repentance}}
\addcontentsline{toc}{section}{On Faith and Repentance}

\begin{enumerate}
\def\labelenumi{\arabic{enumi}.}
\setcounter{enumi}{15}
\item
  God offers the free gift of forgiveness for all sinful corruptions of sexuality through repentance and faith in Jesus. As the household of faith, the church's duty and privilege is to proclaim this repentance and forgiveness. The church is also to seek the healing of those harmed by these sins and to assist parents in training boys to be men and girls to be women.47
\item
  The church has sinned and failed in many ways to obey God's commands of sexuality, even teaching many sexual sins and errors as though they were virtues and truths. This unfaithfulness and the fact that sinners on the road of repentance and faith in Jesus Christ often enter the church bound in deep patterns of sexual sin mean that repentance for sexual sinners may be a gradual and complicated process. Because of this, wise and compassionate pastoral discretion is necessary to practically apply the Bible's teaching on sexuality.
\end{enumerate}

\hypertarget{afterword}{%
\section*{Afterword}\label{afterword}}
\addcontentsline{toc}{section}{Afterword}

This Declaration does not exhaust the extent of our beliefs and practices. The Bible itself, as the inspired and infallible Word of God that speaks with final authority concerning truth, morality, and the proper conduct of mankind, is the sole and final source of authority for all that we believe and do. For purposes of Evangel Presbytery's faith, doctrine, practice, policy, and discipline, the membership of the Presbytery (as constituted under BCO 15.1) is the Presbytery's final interpretive authority on the Bible's meaning and application, including for any purpose under BCO 15.8. As the practical outworking of the beliefs set forth in this Declaration, Evangel Presbytery has adopted policies relating to marriage and sexuality in BCO 71 and BCO 72.

\begin{center}\rule{0.5\linewidth}{0.5pt}\end{center}

\hypertarget{notes}{%
\section*{Notes}\label{notes}}
\addcontentsline{toc}{section}{Notes}

1. Genesis 5:2. Unless otherwise indicated, all Scripture references are to the New American Standard Bible (1995).

2. See Genesis 2:7.

3. See Genesis 2:20--22. See also 1~Corinthians 11:11--12: ``However, in the Lord, neither is woman independent of man, nor is man independent of woman. For as the woman originates from the man, so also the man has his birth through the woman; and all things originate from God.''

4. Genesis 2:23.

5. ``For man does not originate from woman, but woman from man; for indeed man was not created for the woman's sake, but woman for the man's sake.'' 1~Corinthians 11:8--9.

6. ``And the LORD God said, It is not good that the man should be alone; I will make him an help meet for him.'' Genesis 2:18 (KJV).

7. See Ephesians 5:25--33.

8. ``But I do not allow a woman to teach or exercise authority over a man, but to remain quiet. For it was Adam who was first created, and then Eve. And it was not Adam who was deceived, but the woman being deceived, fell into transgression.'' 1~Timothy 2:12--14.

9. ``For a man ought not to have his head covered, since he is the image and glory of God; but the woman is the glory of man.'' 1~Corinthians 11:7.

10. See Ephesians 5:25, 28; cf.~Colossians 3:19.

11. See Ephesians 5:22, 24; cf.~Colossians 3:18 and Titus 2:5.

12. ``God created man in His own image, in the image of God He created him; male and female He created them.'' Genesis 1:27. See also 1~Peter 3:7: ``\ldots you wives, be submissive to your own husbands. \ldots{} You husbands in the same way, live with your wives in an understanding way, as with someone weaker, since she is a woman; and show her honor as a fellow heir of the grace of life, so that your prayers will not be hindered.''

13. See Psalm 139:13--16; Genesis 5:2.

14. Because all truth is God's truth and God is not a man that He should lie (Numbers 23:19), what God reveals through His creation never contradicts what He reveals in His Word. Thus, genetics and other scientific disciplines, when not abused or corrupted for sinful purposes, can declare truth. It is proper then to recognize that a man conceived and born a man is genetically male, and a woman conceived and born a woman is genetically female.

15. This is not to address the extremely rare case of intersex children born with a variety of conditions, including: Not XX and Not XY, Hypospadias, Androgen Insensitivity Syndrome, Ovotestes, etc. ``Intersex'' is a medical diagnosis of atypical male or female anatomies not to be confused with those born with typical male or female anatomies who claim a ``transgender'' or ``transsexual'' identity. The suffering of those born with physical anomalies must not be used to justify the rebellion of those who repudiate the sex God made them. All such abnormalities and genetic deformities along with illness of any kind, pain, suffering, and death itself, are a result of the Fall as described in Genesis 3, and pass through God's sovereign hand as He gives to each man both good and adversity. See Exodus 4:11; Job 2:9--10; John 9:1--7.

16. ``Only, as the Lord has assigned to each one, as God has called each, in this manner let him walk.'' 1~Corinthians 7:17.

17. ``A woman shall not wear man's clothing, nor shall a man put on a woman's clothing; for whoever does these things is an abomination to the LORD your God.'' Deuteronomy 22:5. See also 1~Corinthians 6:9--10: ``Or do you not know that the unrighteous will not inherit the kingdom of God? Do not be deceived; neither fornicators, nor idolaters, nor adulterers, nor effeminate, nor homosexuals, nor thieves, nor the covetous, nor drunkards, nor revilers, nor swindlers, will inherit the kingdom of God.''

18. See 1~Corinthians 6:9--10.

19. See Genesis 3:20.

20. See Ephesians 5:25.

21. See Deuteronomy 22:5. In the text, ``man's clothing''---or, in the KJV, ``that which pertaineth to a man''---refers to the clothing and weapons worn by men for combat.

22. See 1~Peter 3:7 on the comparative weakness of women. Jeremiah 50:37 and Nahum 3:13 both make references to men ``becoming women'' as descriptive of their vulnerability to attack.

23. ``So then, if while her husband is living she is joined to another man, she shall be called an adulteress; but if her husband dies, she is free from the law, so that she is not an adulteress though she is joined to another man.'' Romans 7:3.

24. ``You shall not commit adultery.'' Exodus 20:14. See also 1~Timothy 3:2: ``An overseer, then, must be above reproach, the husband of one wife.''

25. ``But from the beginning of creation, God MADE THEM MALE AND FEMALE. FOR THIS REASON A MAN SHALL LEAVE HIS FATHER AND MOTHER, AND THE TWO SHALL BECOME ONE FLESH; so they are no longer two, but one flesh. What therefore God has joined together, let no man separate.'' Mark 10:6--9. See also 1~Corinthians 6:9--10.

26. See Genesis 2:18--24.

27. ``Husbands, love your wives, just as Christ also loved the church and gave Himself up for her, so that He might sanctify her, having cleansed her by the washing of water with the word, that He might present to Himself the church in all her glory, having no spot or wrinkle or any such thing; but that she would be holy and blameless. So husbands ought also to love their own wives as their own bodies. He who loves his own wife loves himself; for no one ever hated his own flesh, but nourishes and cherishes it, just as Christ also does the church, because we are members of His body. FOR THIS REASON A MAN SHALL LEAVE HIS FATHER AND MOTHER AND SHALL BE JOINED TO HIS WIFE, AND THE TWO SHALL BECOME ONE FLESH. This mystery is great; but I am speaking with reference to Christ and the church.'' Ephesians 5:25--32.

28. ``For the husband is the head of the wife, as Christ also is the head of the church, He Himself being the Savior of the body.'' Ephesians 5:23.

29. Ephesians 5:25.

30. ``Wives, be subject to your own husbands, as to the Lord. For the husband is the head of the wife, as Christ also is the head of the church, He Himself being the Savior of the body. But as the church is subject to Christ, so also the wives ought to be to their husbands in everything.'' Ephesians 5:22--24.

31. Westminster Confession of Faith, ch.~24 (``Of Marriage and Divorce''), para. 2. Westminster Confession of Faith hereinafter abbreviated as ``WCF.''

32. ``When the woman saw that the tree was good for food, and that it was a delight to the eyes, and that the tree was desirable to make one wise, she took from its fruit and ate; and she gave also to her husband with her, and he ate. Then the eyes of both of them were opened, and they knew that they were naked; and they sewed fig leaves together and made themselves loin coverings. They heard the sound of the LORD God walking in the garden in the cool of the day, and the man and his wife hid themselves from the presence of the LORD God among the trees of the garden. Then the LORD God called to the man, and said to him, `Where are you?' He said, `I heard the sound of You in the garden, and I was afraid because I was naked; so I hid myself.' And He said, `Who told you that you were naked? Have you eaten from the tree of which I commanded you not to eat?' The man said, ``The woman whom You gave to be with me, she gave me from the tree, and I ate.''' Genesis 3:6--12.

33. ``He said to them, `Because of your hardness of heart Moses permitted you to divorce your wives; but from the beginning it has not been this way.'\,'' Matthew 19:8.\\
\hspace*{0.333em}\hspace*{0.333em}\hspace*{0.333em}\hspace*{0.333em}``For truly I say to you, until heaven and earth pass away, not the smallest letter or stroke shall pass from the Law until all is accomplished.'' Matthew 5:18.

34. See 1~Corinthians 7:12--13, 15--16. Abandonment or ``wilful desertion'' (WCF 24.6) is not limited only to cases of one spouse's unjustified departure and refusal to be reconciled to the abandoned or injured spouse. A state of willful desertion also exists if the offending party's conduct is so egregious that the injured party is forced to leave the marital home and reconciliation is impossible due to the nature or seriousness of the sin and the offending party's persistent impenitence.

35. ``It was said, `Whoever sends his wife away, let him give her a certificate of divorce'; but I say to you that everyone who divorces his wife, except for the reason of unchastity, makes her commit adultery; and whoever marries a divorced woman commits adultery.'' Matthew 5:31--32. See also WCF 24.6.

36. ``But I say to you that everyone who looks at a woman with lust for her has already committed adultery with her in his heart.'' Matthew 5:28.

37. ``Finally then, brethren, we request and exhort you in the Lord Jesus, that as you received from us instruction as to how you ought to walk and please God (just as you actually do walk), that you excel still more. For you know what commandments we gave you by the authority of the Lord Jesus. For this is the will of God, your sanctification; that is , that you abstain from sexual immorality; that each of you know how to possess his own vessel in sanctification and honor, not in lustful passion, like the Gentiles who do not know God; and that no man transgress and defraud his brother in the matter because the Lord is the avenger in all these things, just as we also told you before and solemnly warned you . For God has not called us for the purpose of impurity, but in sanctification. So, he who rejects this is not rejecting man but the God who gives His Holy Spirit to you.'' 1~Thessalonians 4:1--8.\\
\hspace*{0.333em}\hspace*{0.333em}\hspace*{0.333em}\hspace*{0.333em}``But I say, walk by the Spirit, and you will not carry out the desire of the flesh.'' Galatians 5:16.

38. ``Marriage is to be held in honor among all, and the marriage bed is to be undefiled; for fornicators and adulterers God will judge.'' Hebrews 13:4.

39. ``You shall not commit adultery.'' Exodus 20:14.

40. ``And He answered and said, `Have you not read that He who created them from the beginning MADE THEM MALE AND FEMALE, and said, ``FOR THIS REASON A MAN SHALL LEAVE HIS FATHER AND MOTHER AND BE JOINED TO HIS WIFE, AND THE TWO SHALL BECOME ONE FLESH''? So they are no longer two, but one flesh. What therefore God has joined together, let no man separate.'\,'' Matthew 19:4--6.\\
\hspace*{0.333em}\hspace*{0.333em}\hspace*{0.333em}\hspace*{0.333em}``An overseer, then, must be above reproach, the husband of one wife\ldots{}'' 1~Timothy 3:2.

41. See Leviticus 20:11--12, 14, 17, 19--21.

42. ``He said to His disciples, `It is inevitable that stumbling blocks come, but woe to him through whom they come! It would be better for him if a millstone were hung around his neck and he were thrown into the sea, than that he would cause one of these little ones to stumble.'\,'' Luke 17:1--2.

43. ``You shall not lie with a male as one lies with a female; it is an abomination.'' Leviticus 18:22.

44. ``Also you shall not have intercourse with any animal to be defiled with it, nor shall any woman stand before an animal to mate with it; it is a perversion.'' Leviticus 18:23.

45. See Jeremiah 17:9.

46. ``However, because by this deed you have given occasion to the enemies of the LORD to blaspheme, the child also that is born to you shall surely die.'' 2~Samuel 12:14. This text about God's discipline on King David for his adultery with Bathsheba demonstrates dishonor coming to God's holy name and suffering coming to the members of a man's family---both caused by the sin of the father.
``For `THE NAME OF GOD IS BLASPHEMED AMONG THE GENTILES BECAUSE OF YOU,' just as it is written.'' Romans 2:24.

47. ``Such were some of you; but you were washed, but you were sanctified, but you were justified in the name of the Lord Jesus Christ and in the Spirit of our God.'' 1~Corinthians 6:11.\\
\hspace*{0.333em}\hspace*{0.333em}\hspace*{0.333em}\hspace*{0.333em}``The Spirit of the LORD God is upon me, / Because the LORD has anointed me / To bring good news to the afflicted; / He has sent me to bind up the brokenhearted, / To proclaim liberty to captives / And freedom to prisoners.'' Isaiah 61:1.\\
\hspace*{0.333em}\hspace*{0.333em}\hspace*{0.333em}\hspace*{0.333em}``A BATTERED REED HE WILL NOT BREAK OFF, AND A SMOLDERING WICK HE WILL NOT PUT OUT, UNTIL HE LEADS JUSTICE TO VICTORY.'' Matthew 12:20.\\
\hspace*{0.333em}\hspace*{0.333em}\hspace*{0.333em}\hspace*{0.333em}As one example of scriptural instructions given to men as distinct from those given to women, see Titus 2:2ff.: ``Older men are to be temperate, dignified, sensible, sound in faith, in love, in perseverance. Older women likewise are to be reverent in their behavior, not malicious gossips nor enslaved to much wine, teaching what is good, so that they may encourage the young women to love their husbands, to love their children, to be sensible, pure, workers at home, kind, being subject to their own husbands, so that the word of God will not be dishonored. Likewise urge the young men to be sensible\ldots{}''

\hypertarget{declaration-of-doctrine-on-abortion}{%
\chapter*{Declaration of Doctrine on Abortion}\label{declaration-of-doctrine-on-abortion}}
\addcontentsline{toc}{chapter}{Declaration of Doctrine on Abortion}

\hypertarget{introduction-1}{%
\section*{Introduction}\label{introduction-1}}
\addcontentsline{toc}{section}{Introduction}

When the history of our time is written, it will be a record of bloodshed on a scale previously unimaginable across the history of mankind. This bloodshed began with two World Wars in which combatant fatalities were numbered in the tens of millions. By the end of the Second World War, militaries of both sides had turned to targeting their enemy's civilians so that combatant fatalities were surpassed by non-combatant fatalities. The growth in bloodshed then turned from soldiers killing soldiers, to soldiers killing civilians, to Communist rulers killing their own citizens. Russia's Joseph Stalin and China's Mao Zedong slaughtered well over 100 million of their own people.

But it was not until bloodshed entered the home, and fathers and mothers killed their own offspring, that the bloody count exploded. Five decades into abortion's slaughter, the death toll now is in the billions, and growing. Unmoored from the restraints of God's Law, man has devised ever more sophisticated ways to oppress and destroy the weak, aided in this by technology that magnifies his corruption. From the proliferation of contraceptive1 (and often abortifacient2) drugs, to the denial of life at the moment of conception, to the mass use of abortifacient drugs (e.g., RU-486), to the manipulation and destruction of human life by reproductive technology (i.e., in-vitro fertilization), and now to sex-selective abortion, the history of the twentieth and twenty-first centuries is of man using his ingenuity to advance his lusts: ``Truly, this only I have found: That God made man upright, But they have sought out many schemes'' (Eccl. 7:29).

Until the twentieth century, the Church's condemnation of such wickedness had been united and unanimous. Beginning in the 1930s, historic Protestant denominations began to soften and remove their long-standing opposition to contraception. So when the offensive shifted to its sister fruit, abortion, Protestants no longer had a framework for interpreting (and then opposing) this evil, and denomination after denomination were either supportive or silent. In this context, the bloodshed grew, not only among unbelievers, but among conservative Christians as well.

There were, eventually, also victories. Starting in the early 1980s, Protestants recovered some of their will to fight this evil, and persistent work over fifty years yielded a substantial victory in the Supreme Court's 2022 \emph{Dobbs} ruling. Yet even the overturning of \emph{Roe v. Wade} is but a partial victory---especially as it has revealed even among conservative Christians the lack of will to end abortion. For even as we, God's people, have opposed abortion, the blood has continued to stain our country, our streets, even our homes.

Yet God is faithful, and He has promised to purify His Church with the truth. So it is our desire to repent of the Church's failure in the matter of abortion, and to read with new eyes and new hearts God's commands concerning life, death, and children. The statements below are not everything we teach (or that must be taught) on these matters, but, like Augustine, we speak even in part so that we may not be entirely silent.

\hypertarget{statement-1}{%
\section{Statement 1}\label{statement-1}}

\begin{enumerate}
\def\labelenumi{\arabic{enumi}.}
\tightlist
\item
  The Protestant Church's weakness in anthropology, uncritical embrace of contraception, and tacit acceptance of abortion are scandals, and the products of doctrinal passivity and retreat. From these we must repent, and must recover the biblical and historical doctrines of anthropology to address the evils facing us today.
\end{enumerate}

\hypertarget{statement-2}{%
\section{Statement 2}\label{statement-2}}

\begin{enumerate}
\def\labelenumi{\arabic{enumi}.}
\setcounter{enumi}{1}
\tightlist
\item
  God created man in His own image---male and female He created them. The life of man on this earth begins at the moment of conception, which is the joining of the father's sperm with the mother's egg. From that point onward he bears the image of God, is the pinnacle of God's Creation, and possesses all the rights of personhood.
\end{enumerate}

\hypertarget{statement-3}{%
\section{Statement 3}\label{statement-3}}

\begin{enumerate}
\def\labelenumi{\arabic{enumi}.}
\setcounter{enumi}{2}
\tightlist
\item
  Abortion as condemned by Scripture and the Presbytery means the taking of a human life3 at any point from conception to birth.4 It is a grave sin in the sight of God, and is opposed in both Old and New Testaments, as well as by all branches of the Church throughout all time.5 Further, as the murder of an unborn child, abortion is a particularly heinous offense, and Scripture condemns it in the strongest terms.6
\end{enumerate}

\hypertarget{statement-4}{%
\section{Statement 4}\label{statement-4}}

\begin{enumerate}
\def\labelenumi{\arabic{enumi}.}
\setcounter{enumi}{3}
\tightlist
\item
  Abortion is an offense against reason, society, the laws of nature, the woman's conscience, and most of all, the Law of God, whose Sixth Commandment forbids murder and commands the protection and preservation of human life.7 Man's law can never repeal, abrogate, or supersede God's Law, which is perfect.
\end{enumerate}

\hypertarget{statement-5}{%
\section{Statement 5}\label{statement-5}}

\begin{enumerate}
\def\labelenumi{\arabic{enumi}.}
\setcounter{enumi}{4}
\tightlist
\item
  The taking of a human life is justifiable only in limited circumstances (self-defense, capital punishment, and just war) to protect oneself or another man, woman, or child from the attack of an aggressor. Abortion has no commonality with any of these; it is itself an attack upon the defenseless.
\end{enumerate}

\hypertarget{statement-6}{%
\section{Statement 6}\label{statement-6}}

\begin{enumerate}
\def\labelenumi{\arabic{enumi}.}
\setcounter{enumi}{5}
\tightlist
\item
  Many forms of birth control operate not only by preventing conception but also by preventing the implantation of a new embryo in the mother's womb. This is especially true of intrauterine devices and hormonal contraceptive methods, including the birth control pill and the morning-after pill. By denying sustenance to a new life, they cause the death of an unborn child at one of its earliest stages. This is the commission of an unjustifiable homicide.
\end{enumerate}

\hypertarget{statement-7}{%
\section{Statement 7}\label{statement-7}}

\begin{enumerate}
\def\labelenumi{\arabic{enumi}.}
\setcounter{enumi}{6}
\tightlist
\item
  God created marriage to bring forth life, commanding Adam and Eve in the state of perfection to ``be fruitful and multiply.'' All creation bears witness to this principle, and, still today, all husbands and wives are to obey this Creation ordinance.8 Moreover, throughout history both pagans and Christians have recognized an essential connection between abortion and contraception, and condemnation of the one has gone hand in hand with that of the other. Thus, while we may distinguish (non-abortifacient) contraception from abortion, we cannot wholly separate the two. The sexual union between man and woman is itself theological revelation about marriage, Christ, and the Church, and the unity of marriage's procreative and unitive purposes is fundamental to God's institution of marriage.
\end{enumerate}

\hypertarget{statement-8}{%
\section{Statement 8}\label{statement-8}}

\begin{enumerate}
\def\labelenumi{\arabic{enumi}.}
\setcounter{enumi}{7}
\tightlist
\item
  Nevertheless, in this fallen world situations arise in which this Creation ordinance collides with other commands of God. By God's grace, such situations are exceptional and generally temporary; often, they call for wise and sensitive pastoral care. Any decision to frustrate natural conception must be taken recognizing God's command of fruitfulness as good, not in fearful rejection of it---and with a firm resolve not to use any contraceptive method with the smallest agency of killing the unborn child. In the pattern of our lives, we (and all men) are to show that our marital unions are marked by openness to the children God has designed for us, and by our willingness to nourish them with life.
\end{enumerate}

\hypertarget{statement-9}{%
\section{Statement 9}\label{statement-9}}

\begin{enumerate}
\def\labelenumi{\arabic{enumi}.}
\setcounter{enumi}{8}
\tightlist
\item
  While science has revealed much wonder in how God fashions us in the womb, Christians must not reduce this mystery to the mechanical and empirical. Only a glimpse of this process lies before our eyes; God alone beholds our frame as we are ``made in secret and skillfully wrought in the lowest parts of the earth'' (Ps. 139:15). Scientific and medical advancements are the Lord's gifts to mitigate the sorrows of this world, including infertility, yet Christians must tread with care in all such areas, recognizing the secret things of the Lord (Deut. 29:29). It belongs to God alone to join soul and body, and whatever technological or medical developments promise to heal the pain of infertility must be subjected to careful study before any of them is used. Nor must we ever treat the fruit of the womb as simply a DNA-bearing zygote instead of an immortal bearer of God's image, or profane the sacred unitive context in which God has designed that life to be nourished.
\end{enumerate}

\hypertarget{statement-10}{%
\section{Statement 10}\label{statement-10}}

\begin{enumerate}
\def\labelenumi{\arabic{enumi}.}
\setcounter{enumi}{9}
\tightlist
\item
  As but one example of this, in-vitro fertilization as typically practiced9 is illicit and immoral. By design, it separates the unitive and procreative functions of marital intercourse, and supplants this unitive act with mechanical and technological production. But even worse, it results in the abuse and death of millions of preborn children. For Christians to foster and subsidize an industry of mechanized reproduction is a scandal. Yes, infertility is a grievous sorrow to bear, and medical assistance should of course be sought to alleviate it. But God's people seek the Lord, first, knowing He is pleased to give good gifts to His children. And if, despite patience, prayer, and lawful medical intervention, God's answer is ``no,'' Christians receive this by faith, rather than by striving to engineer fertility no matter the cost. He is the One Who opens and closes the womb, and by His design our children are conceived in the secret places. Finally, let it be said clearly that ``You shall not murder'' is true and binding even if murdering provides a couple with a child of their own.
\end{enumerate}

\hypertarget{statement-11}{%
\section{Statement 11}\label{statement-11}}

\begin{enumerate}
\def\labelenumi{\arabic{enumi}.}
\setcounter{enumi}{10}
\tightlist
\item
  Rape and incest, while serious sins of their own, do not justify abortion. Rather, abortion in these circumstances magnifies and perpetuates violence by inflicting it on the innocent unborn child, punishing him for the crime of another.10
\end{enumerate}

\hypertarget{statement-12}{%
\section{Statement 12}\label{statement-12}}

\begin{enumerate}
\def\labelenumi{\arabic{enumi}.}
\setcounter{enumi}{11}
\tightlist
\item
  Christians should not be misled by specious arguments for abortion on the basis of ``the health of the mother,'' which involve not a genuine ethical dilemma, but a category so broad as to regard any physical or psychological impediment as grounds for the murder of a child.11
\end{enumerate}

\hypertarget{statement-13}{%
\section{Statement 13}\label{statement-13}}

\begin{enumerate}
\def\labelenumi{\arabic{enumi}.}
\setcounter{enumi}{12}
\tightlist
\item
  Many of us among the people of God have killed our children by abortion. The women and men guilty of this sin should be exhorted by their pastors to faith and repentance that neither disregards their moral agency nor removes their hope in the grace of God. Freedom from the bloodguilt of abortion comes not by claiming victimhood but by acknowledging guilt and fleeing to the cross of Christ for His forgiveness.
\end{enumerate}

\hypertarget{statement-14}{%
\section{Statement 14}\label{statement-14}}

\begin{enumerate}
\def\labelenumi{\arabic{enumi}.}
\setcounter{enumi}{13}
\tightlist
\item
  Civil magistrates must use every tool available to end abortion immediately. As murder, abortion is not only a serious sin but a grave crime to be prohibited and punished by the civil government. Although it may take generations to fully eliminate this evil, magistrates must not shirk their responsibility out of cowardice and unfaithfulness. They should remember that they will answer to God.
\end{enumerate}

\hypertarget{statement-15}{%
\section{Statement 15}\label{statement-15}}

\begin{enumerate}
\def\labelenumi{\arabic{enumi}.}
\setcounter{enumi}{14}
\tightlist
\item
  Civil magistrates must temper their zeal with wisdom from on high. In some cases, wisdom will mean supporting a law that would abolish abortion. In other cases, it may require a ruler to prioritize an incremental strategy offering the most long-term benefits for ending the bloodshed. If the immediate end of abortion is not possible, an incremental approach is not necessarily weak or compromised.12
\end{enumerate}

\hypertarget{statement-16}{%
\section{Statement 16}\label{statement-16}}

\begin{enumerate}
\def\labelenumi{\arabic{enumi}.}
\setcounter{enumi}{15}
\tightlist
\item
  Christians must prioritize abortion as an electoral issue. While Christian principles on abortion are not inherently partisan, Christians must not let fear of appearing partisan discourage them from voting in a Christian manner. To be sure, the commitments of a particular politician or party may be illusory and cynical, with no desire to end the slaughter. But worse is the party that has made legal abortion its reason to exist. We must both oppose the hypocrisy of the former and abominate the bloodlust of the latter.
\end{enumerate}

\hypertarget{statement-17}{%
\section{Statement 17}\label{statement-17}}

\begin{enumerate}
\def\labelenumi{\arabic{enumi}.}
\setcounter{enumi}{16}
\tightlist
\item
  Church leaders must regularly and with authority teach against abortion and lead their flocks to love the fruit of the womb. They should also equip their congregants to battle abortion as their station and calling in life allow. In doing so, care should be taken that any division that follows is in fact necessary. Regarding opposition to abortion, good men will differ on method and timing, and comrades in arms will often not see strategy and tactics in the same way. We must grow in our discernment so we can distinguish between reformers and schismatics.
\end{enumerate}

\hypertarget{statement-18}{%
\section{Statement 18}\label{statement-18}}

\begin{enumerate}
\def\labelenumi{\arabic{enumi}.}
\setcounter{enumi}{17}
\tightlist
\item
  In abhorring abortion, Christians must also be discerning when facing situations that defy easy ethical answers. Sometimes, mothers, fathers, and physicians face difficult decisions where a child may die as a result of otherwise life-saving medical treatment. Survival of the mother and survival of her child may appear mutually exclusive (e.g., in chemotherapy or radiation treatment), and yet here, too, the physician must attend to both patients, disregarding neither. Parents and physician must strive for the survival of both mother and child, desiring to live with a clean conscience before the Lord. Yet while a man plans his way, ``the Lord directs his steps'' (Prov. 16:9), and if the medical decisions result in the death of the child (or the mother), all parties can take comfort in the knowledge that this was in no way their intent.13 They acted wisely, yet God's decree was that one would live and the other would die. ``The Lord giveth, and the Lord taketh away; blessed be the Name of the Lord'' (Job 1:21).
\end{enumerate}

\hypertarget{statement-19}{%
\section{Statement 19}\label{statement-19}}

\begin{enumerate}
\def\labelenumi{\arabic{enumi}.}
\setcounter{enumi}{18}
\tightlist
\item
  Pastors and elders should be especially tender when their sheep suffer the loss of their little ones. The death of a child, born or unborn, is a severe dispensation of the Lord's providence, and the grieving parents should be the recipients of their shepherds' and church members' ongoing guidance and care.
\end{enumerate}

\hypertarget{statement-20}{%
\section{Statement 20}\label{statement-20}}

\begin{enumerate}
\def\labelenumi{\arabic{enumi}.}
\setcounter{enumi}{19}
\tightlist
\item
  We who are God's people must love our children---those who are born, and those yet to be born---and we must trust the Lord to provide all that we need, both for us and our children. And not only our own children---we must also love all children, those of our covenant communities, but also those of the unbelievers outside it. For our own children have been spared their fate only by God's mercy. The notion of a surplus population is not of Christ, and let no Christian be guilty of such thoughts toward those who bear the image of God.
\end{enumerate}

\hypertarget{statement-21}{%
\section{Statement 21}\label{statement-21}}

\begin{enumerate}
\def\labelenumi{\arabic{enumi}.}
\setcounter{enumi}{20}
\tightlist
\item
  We must also uphold and commend the work of adoption for those who are called to it. Rescuing children both inside the womb and outside it has always been a mark of God's people. The early Christians knew that their identity and communion with Christ was bound up with the little babies left to die, and that to care for them was to honor their Lord Jesus and His birth.
\end{enumerate}

\hypertarget{statement-22}{%
\section{Statement 22}\label{statement-22}}

\begin{enumerate}
\def\labelenumi{\arabic{enumi}.}
\setcounter{enumi}{21}
\tightlist
\item
  God in His providence will end abortion according to His own counsel---``There are many plans in a man's heart; nevertheless the counsel of the Lord---that will stand'' (Proverbs 19:21). Yet the end of abortion will come only with the gospel, and with the Church leading the world to embrace fruitfulness. Until that day, we labor on, embracing the fruit of the womb, and loving children---and so proving ourselves true sons of our heavenly Father.
\end{enumerate}

\hypertarget{statement-23}{%
\section{Statement 23}\label{statement-23}}

\begin{enumerate}
\def\labelenumi{\arabic{enumi}.}
\setcounter{enumi}{22}
\tightlist
\item
  In proclaiming these truths, we stand not alone, but in the company of our Lord and of His Church throughout the ages, pronouncing the grave evil and crime of abortion, but also the absolute blessing of children to our families, our churches, and our land. To bring forth children in an evil day, and to call all men to love what God has lavishly bestowed, is to hope for a day we cannot see, and with patience to wait for it. And so it is in this hope that we persevere---in unity with Apostles and Prophets, the holy Fathers of old, the Magisterial Reformers, and the faithful churchmen of all the ages, until, finally, our eyes may see the day when this mighty scourge of evil shall speedily pass away.
\end{enumerate}

In the name of the Father, the Son, and the Holy Spirit, Amen.

\begin{center}\rule{0.5\linewidth}{0.5pt}\end{center}

\hypertarget{notes-1}{%
\section{Notes}\label{notes-1}}

1. In saying this we do not condemn all use of contraception. Although most contraception today is not morally justified, occasions do arise where it may be warranted. In such cases, great care must be taken to ensure that whatever method is used does not end a newly conceived life in the womb (see Statement~6 for more on this). Such methods (generally barrier) are not, however, the most common methods of contraception used today. Beyond this, the unrestrained use of contraception is symptomatic of the selfishness that despises both a woman's unique gift of childbearing and the fruit of that gift, children---which leads to the bloodshed we condemn here.

2. That is, ``abortion-causing'' drugs, whether their operation is direct (e.g., RU-486) or indirect (oral contraceptives such as ``The Pill''); see Statement~6 below.

3. See Statement~5 for explanation of when the taking of human life is justifiable.

4. See Statement~18 for discussion of challenging situations which may involve actions that unintentionally result in the death of an unborn child.

5. For a sampling of the unanimity of the Church throughout the ages on this topic, see ``The Witness of Church History'' in
Evangel Presbytery's Abortion and the Church (\url{https://abortion.evangelpresbytery.com/the-witness-of-church-history.html\#the-witness-of-church-history}).

6. The condemnation of the murder of children is ubiquitous across the Old Testament, and these condemnations are pronounced against the Canaanites and the sons of Israel alike. Molech worship required that a child be placed in the mouth of the god as a burnt offering. This is a sin so heinous to God that it is the only evil said never to have entered His mind (Jer. 32:35).

7. Consider the Westminster Larger Catechism: ``Q135: What are the duties required in the sixth commandment?'' Among the duties required are ``all careful studies, and lawful endeavors, to preserve the life of ourselves and others by resisting all thoughts and purposes, subduing all passions, and avoiding all occasions, temptations, and practices, which tend to the unjust taking away the life of any.''

8. This does not mean, of course, that God will grant every man and wife a child. God opens and closes the womb as He sees fit, yet His Creation ordinance still stands for all men across time. See Statement 10 for more on a godly approach to infertility.

9. Almost always, it is intrinsic to IVF to produce sixteen embryos to ensure the survival of only one---i.e., a 94\% fatality rate. For more, see ``In Vitro Fertilization: Babies in the Fridge'' in Abortion and the Church (\url{https://abortion.evangelpresbytery.com/abortions-consequences.html\#in-vitro-fertilization-babies-in-the-fridge}).

10. See also ``Rape and Incest'' in Abortion and the Church (\url{https://abortion.evangelpresbytery.com/dealing-with-commonjustifications-for-abortion.html}).

11. For discussion of difficult cases involving the life of the mother, see Statement 18 in this document. For more on ``health of the mother'' arguments for abortion, see ``Health of the Mother'' in Abortion and the Church (\url{https://abortion.evangelpresbytery.com/dealing-with-common-justifications-for-abortion.html\#health-of-the-mother}).

12. For an in-depth discussion on these issues, see the entire section ``The Duty of Civil Authorities'' in Abortion and the Church (\url{https://abortion.evangelpresbytery.com/the-duty-of-civil-authorities.html\#the-duty-of-civil-authorities}).

13. Such a case, where medical treatment causes the death of a child in spite of the physician's best efforts, is the very opposite of abortion. When a mother pays to abort her child it is almost never to save her own life or health, but rather to maintain her lifestyle and future plans.

\hypertarget{updates}{%
\chapter*{Updates}\label{updates}}
\addcontentsline{toc}{chapter}{Updates}

\emph{Significant changes to the Doctrinal Declarations of the Presbytery will be listed here. For a detailed diff hosted at Github, \href{https://github.com/Evangel-Presbytery/evangel-ddp}{click here}.}

\begin{itemize}
\tightlist
\item
  Original Version adopted by Evangel Presbytery on June 5, 2025. First published ------------.
\end{itemize}

\end{document}
